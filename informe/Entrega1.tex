\documentclass[a4paper, 10pt]{article}

\usepackage{hyperref}
\usepackage{amsmath}
\usepackage{amsfonts}
\usepackage{enumerate}
\usepackage{tkz-fct}
\usepackage{graphicx}
\usepackage{subcaption}
\usepackage{geometry}

% Separaciones
\setlength{\parindent}{0mm}
\setlength{\parskip}{1mm}
\geometry{a4paper,margin=1.3in}

% Comandos propios
\newcommand{\atx}{\texttt{Asterix} }

% Title 
\title{\textsc{Asterix} \\ \Large\textit{Impertative programming language}}
\author{Sergio Domínguez Cabrera y Enrique Carro Garrido}
\date{}

\begin{document}
    \maketitle   
    
    % Que empiece la fiesta
    \begin{figure}[h]
        \centering
        \includegraphics{Abra_porteurs_SP-400x613.png}
    \end{figure}

    % Frase de asterix
    \begin{verse}
        \raggedleft{\textit{
        No, Obélix. Tú no tendrás la poción mágica. \\
        }
        \textsc{Asterix el Galo}
        \footnote{ Esta es una frase famosa que se repite en toda la saga.
        Cuando era niño, Obélix cayó en un caldero de poción mágica, cosa que
        ocasionó efectos a perpetuidad. Sin embargo, eso no le impide a Obélix
        intentar obtener una poción mágica cada vez que se presenta la
        oportunidad. }
        }
    \end{verse}
    \vspace{3pt}

    % Abstract
    \begin{small}
        \hrule
        \vspace{\baselineskip}
        En el presente texto se realiza una descripción detallada del lenguaje
        de programación imperativo \atx. 
        \vspace{\baselineskip}\hrule
    \end{small}

    \section*{Introducción a Asterix}
    \atx nace de la idea de ser un lenguaje de programación de paradigma 
    imperativo, con tipado estático y portable. Los programas escritos en \atx tienen que
    tener la extensión .atx y la codificación de los ficheros .atx es UTF-8.
    ¿Pero qué sería \atx sin \texttt{Obelix}? El compañerismo de la serie 
    se traslada al lenguaje de forma que el compilador del mismo recibe el
    nombre de este gran compañero. 

    \section*{Tipos básicos y variables}
    Por defecto, \atx es capaz de distinguir 4 tipos básicos:
    \begin{itemize}
        \item \textsc{Integerix}: Representa el subconjunto finito de los números
            enteros que pueden ser expresados con 4 bytes. Para declarar una
            variable con este tipo hay que usar la palabra reservada \textsf{intix}.
        \item \textsc{Booleanix}: Representa los valores de lógica binaria, esto
            es dos valores, que en nuestro lenguaje representamos como
            \textsf{galo} (true) y \textsf{romano} (false). Empleamos un byte
            para almacenar estos valores. Para declarar una variable con este
            tipo hay que usar la palabra reservada \textsf{boolix}.
        \item \textsc{Floatix}: Representa el subconjunto finito de los números
            racionales que pueden ser expresados con 8 bytes. Para declarar una
            variable con este tipo hay que usar la palabra reservada \textsf{floatix}.
    \end{itemize}

    \atx permite agrupar estos tipos en \textsc{vectix} que son listas ordenadas
    de elementos de un único tipo (que también pueden ser \textsc{vectix}).
    Para declarar una variable como \textsc{vectix} hay que usar la palabra 
    reservada \textsf{vectix}. Para acceder al elemento \texttt{i} del vectix
    se usa el operador [].

    \atx también soporta la creación de \textsc{pots} \footnote{La marmita o
    \textit{pot} en inglés hace referencia a la marmita a la que se cayó
    Obélix.} que son una agrupación de variables que pueden ser de tipos
    distintos (incluso pueden contener otras \textsc{pots}). Para declarar una
    variable con este tipo hay que usar la palabra reservada \textsf{pot}.

    \subsection*{Declaración de variables}
    Para declarar una variable hay que seguir el siguiente formato: 
    \begin{verbatim}
        tipo identificador (= valor inicial);
    \end{verbatim}
    
    Las variables dentro de un mismo ámbito no pueden tener el mismo identificador. 
    \atx permite también la creación de variables globales. Para declarar una variable como 
    global, se usa la palabra reservada \textsf{global} y se escriben fuera de
    la función main. 

    \subsubsection*{Variables sin inicializar}
    Empleamos el siguiente ejemplo para ilustrar su uso cuando
    las declaramos sin inicializar:

    \begin{verbatim}
    intix num1;
    floatix num2;
    vectix<vectix<intix>[3]>[3] matriz;
    \end{verbatim}

    Estas variables sin inicializar, nosotros las vamos a guardar con un valor.
    Los intix se inicializan con el valor inicial 0, los floatix con 0.0 y los
    charix con el valor inicial romano.
    
    \subsubsection*{Variables inicializadas}
    Empleamos el siguiente ejemplo para visualizar la declaración de variables
    con valor inicial:

    \begin{verbatim}
    floatix num2 = 1.0;
    vectix<intix>[3] numeros = {1,2,3};
    \end{verbatim}


    \section*{Caracteres especiales}
    \atx permite al programador añadir comentarios de una línea al código. Estos
    comentarios son ignorados en el proceso de compilación junto con los espacios,
    tabulaciones y saltos de línea. Los comentarios comienzan siempre con el
    identificador // y se ignora todo lo que haya escrito hasta el siguiente
    salto de línea. Como ya avanzamos en la sección anterior, empleamos el
    caracter \textsc{;} para separar las instrucciones del programa. Mostramos
    ahora unos ejemplos de comentario para ayudar a comprender su uso:

    \begin{verbatim}
    // Esto es un comentario
    intix variable; // Esta variable representa ...
    \end{verbatim}

    \section*{Operadores}
    \atx cuenta con una serie de operadores binarios y unarios que facilitan
    la programación. 
    Escribimos en la siguiente tabla los operadores, su nivel de prioridad y 
    su asociatividad. Los operadores aritméticos solo pueden ser aplicados 
    sobre dos objetos del mismo tipo (y devuelven otro elemento de ese tipo),
    es decir, no podemos operar objetos de diferente tipo. No obstante, estos
    operadores tienen las mismas cualidades en ambos tipos.

    \begin{table}[t]
    \begin{center}
    \begin{tabular}{| c | c | c | c | c |}
    \hline
        Prioridad & Operador & Asociatividad & Aridad & Descripción \\ \hline
        0 & , & Izquierda & Binario & Operador coma \\ \hline
        1 & $=$ & Derecha & Binario & Operador de asignación \\ \hline
        2 & $\|$ & Derecha & Binario & Operador or lógico \\ \hline
        3 & \&\& & Derecha & Binario & Operador and lógico \\ \hline
        4 & == & Derecha & Binario & Operador igual relacional \\ \hline
        4 & != & Derecha & Binario & Operador distinto relacional \\ \hline
        5 & $<=$ & Derecha & Binario & Operador menor o igual relacional \\ \hline
        5 & $>=$ & Derecha & Binario & Operador mayor o igual relacional \\ \hline
        5 & $>$ & Derecha & Binario & Operador mayor relacional \\ \hline
        5 & $<$ & Derecha & Binario & Operador menor relacional \\ \hline
        6 & + & Izquierda & Binario & Suma de dos elementos \\ \hline
        6 & - & Izquierda & Binario & Resta de dos elementos \\ \hline
        6 & \# & Derecha & Binario & Operador concatenación stringix \\ \hline
        7 & * & Izquierda & Binario & Producto de dos elementos \\ \hline
        7 & / & Izquierda & Binario & División de dos elementos \\ \hline
        7 & \% & Izquierda & Binario & Modulo de dos elementos \\ \hline
        7 & $\wedge$ & Derecha & Binario & Potencia de dos elementos \\ \hline
        8 & ! & No asociativo & Unario & Operador not lógico prefijo \\ \hline
        8 & \& & Asociativo & Unario & Paso de elemento por referencia \\ \hline
        9 & . & Izquierda & Binario & Acceso a elemento de un pot \\ \hline
        9 & $[*]$ & Izquierda & Binario & Acceso a elemento de un vectix \\ \hline
        9 & $(*)$ & Izquierda & Binario & Denota precedencia \\ \hline
        

     
    \hline
    \end{tabular}
    \end{center}
    \end{table}

    \section*{Funciones}
    \atx permite dividir el trabajo que hace un programa en tareas más
    pequeñas separadas de la parte principal. Estas tareas son lo que conoceremos
    como pociones. Todo programa escrito en \atx comienza su ejecución en la
    función \textsf{panoramix}\footnote{Panorámix, el druida. Creador de la
    poción mágica, el hombre más sabio del pueblo. En el primer libro, tiene su
    casa al lado de un manantial, y su cabaña tiene un palomar y una gran
    chimenea. Su nombre proviene de panoramique (panorámico). La razón de este
    nombre para el main del programa es ser el creador de las pociones, que son
    las subrutinas en \atx} y es ella la que llama al resto de procedimientos.
    Las funciones siguen el siguiente esquema: 

    \begin{verbatim}
    potion nombre_funcion(arg1 : type1,.., argn : typen) -> typer {
        typer var;
        // Cuerpo de la función
        return var; 
    }
    \end{verbatim}

    \begin{verbatim}
    potion nombre_procedimiento(arg1 : type1,.., argn : typen) {
        // Cuerpo del procedimiento
    }
    \end{verbatim}
    
    Todas las funciones se declaran con la palabra reservada \texttt{poc}. La
    flecha (\texttt{->}) junto con \texttt{typer} indican el tipo del valor que
    devuelve la función, en su ausencia, significa que la función no devuelve
    ningún valor. Las variables \texttt{argi} son los parámetros que recibe la
    función y vienen escritas junto con su tipo \texttt{typei}. Si estamos
    pasando ese argumento por referencia, escribimos \texttt{\&arg}.
    
    \subsection*{Llamadas a función}
    Habíamos definido las funciones para dividir el programa en subrutinas que
    se pueden llamar desde el \textsf{panoramix} o desde otra subrutina.
    
    Las funciones se pueden llamar desde cualquier sitio, teniendo en cuenta
    que deben ser coherentes con su tipo.
    
    Para llamar a una función utilizaremos el siguiente formato:
    \begin{verbatim}
        nombre_funcion(arg1,arg2,...,argn)
    \end{verbatim}
    
    Consideraciones a tener en cuenta:
    \begin{itemize}
        \item Se puede definir una función dentro de otra. Hay que tener en
            cuenta que se ejecutará primero la función más interna.
        \item Los argumentos de la llamada de la función, deben corresponderse
            tanto en tipo, número y orden en el que aparecen.
    \end{itemize}

    \section*{Instrucciones}
    En esta sección presentamos las instrucciones del lenguaje. Algunas de estas
    instrucciones dependen de una condición que ha de ser una sentencia que
    devuelva un valor booleanix, y esta puede ser el valor lógico true (galo),
    si la condición se cumple, o romano si esta no se cumple (false). También
    puede contener el nombre de una variable booleanix, y el valor de la
    expresión dependerá de su contenido. Se debe tener en cuenta que además de
    las variables también puede haber llamadas a pociones que devuelvan un
    valor.

    \subsection*{Instrucciones condicionales}
    \atx admite la estructura de bifurcación condicional, para determinar que
    acciones tomar dada o no cierta condición. La sintaxis que sigue la
    estructura if-then-else es la siguiente:
    
    \begin{verbatim}
    if ( condición ) { 
        // Acciones a realizar si se cumple la condición.
    }
    else {
        // Acciones a realizar si no se cumple la condición.
    }

    // If sin else
    if (condición ) {
        // Acciones a realizar si se cumple la condición.
    }
    \end{verbatim}
    
    Para poder diferenciar unas con otras se obliga que se pongan llaves para
    indicar el ámbito de cada estructura.

    \subsection*{Bucle whilix}
    El bucle whilix es un ciclo repetitivo basado en los resultados
    de una expresión lógica. El propósito es repetir un bloque de código
    mientras una condición se mantenga verdadera. La sintaxis que sigue esta
    estructura es:

    \begin{verbatim}
    whilix ( condición ) {
        // Acciones a realizar mientras se cumpla la condición.
    }
    \end{verbatim}
    
    \subsection*{Bucle forix}
    \atx soporta la estructura \textsf{forix} que permite otra sintaxis para el
    forix más segura a la hora de recorrer vectix y es la siguiente:

    \begin{verbatim}
    forix (type c : vectix<type>) {
        // Bloque de instrucciones
    }
    \end{verbatim}

    En este ejemplo realizamos una vez el bloque de instrucciones por cada
    elemento del vectix que instanciamos como \texttt{c} en cada iteración.
    
    \subsection*{Más instrucciones del lenguaje}
    \atx viene con más instrucciones por defecto:
    \begin{itemize}
        \item \texttt{return X}: Es obligatoria al final de las pociones que 
            devuelven un valor de un cierto tipo. Indica el valor que se devuelve.
    \end{itemize}  
    
    \section*{Entrada y salida}
    
    Un programa \atx puede realizar operaciones de entrada y salida. En esta
    sección supondremos que la entrada proviene del teclado y que las salidas
    se envían a la pantalla de un terminal.
    
    \subsection*{Entrada con tabellae\footnote{Hace referencia a las
    \textit{tabellae ceratae} en latín o tablillas de cera, que eran la
    herramienta que utilizaban los romanos para escribir notas o textos no muy
    largos. Entendemos que la tabellae es realmente la consola y que realmente
    leemos los datos de la tabellae.}}
    
    \textsf{tabellae} es el flujo de entrada estándar. La entrada se hace por
    teclado y para que el programa interprete la salida como un tipo concreto
    \atx tiene distintas instrucciones que las reconoce por separado.
    
    El formato general de todas esas instrucciones son:
    
    \begin{verbatim}
        tabellae( Variable a leer );
    \end{verbatim}
    
    El tipo de la variable  puede ser \textsf{boolix}, \textsf{intix}, o
    \textsf{floatix}. Lo que hace la sentencia anterior es leer un dato
    introducido por teclado, interpretarlo con el tipo correspondiente y
    almacenarlo en la variable. 

    \subsection*{Salida con stilus\footnote{La herramienta que utilizaban los
    romanos para escribir en estas \textit{tabellae}. Eran una especie de
    punzones hechos de hierro, de bronce o de hueso, con un extremo puntiagudo
    y el otro plano o redondeado. Se escribía sobre la capa de cera con el
    extremo puntiagudo y con el otro, si era necesario, se borraba y la tableta
    quedaba lista para volver a escribir. Es el antecedente directo de los
    lápices actuales que llevan una goma de borrar fijada en un extremo.}}
    
     Los valores de variables se pueden enviar a la pantalla empleando
     \textsf{stilus}. El formato general es:
    
    \begin{verbatim}
        stilus(Valor de una variable);
    \end{verbatim}
    
    El valor de variables deben ser variables de tipo \textsf{intix} y \textsf{floatix},
    
    \begin{verbatim}
        intix x = 5;
        stilus (x); // Saca por consola el valor 5
    \end{verbatim}

    \section*{Gestión de errores}
    \atx proporciona una gestión de errores básica indicando la línea y el motivo
    del error. El compilador intentará seguir con el análisis sintántico y semántico
    buscando posibles errores posteriores.

    \section*{Ejemplos de código}
    No sería un lenguaje de programación como
    Tutatis\footnote{\textbf{Teutates}, también llamado \textbf{Tutatis}, es la
    deidad de la unidad tribal masculina del panteón galo, según la antigua
    mitología celta. Hace referencia a la famosa frase que Ásterix repetía una
    y otra vez: "Por Tutatis"} manda si no es capaz de recibir al programador
    con un "Hola mundo", aunque vamos a intercambiar esta frase por otra.
    
    \begin{verbatim}
    potion panoramix() -> intix {
        stilus(1.0);
        return 0;
    }
    \end{verbatim}
    
    Veamos ahora como sería un programa que multiplica dos matrices 3x3.
    \begin{verbatim}
// Programa para multiplicar dos matrices, una es una matriz global, la matriz
// A y la otra la recibimos por consola, la matriz B.

vectix<vectix<intix>[3]>[3] A = {{1, 2, 3}, {1, 2, 3}, {1, 2, 3}};

potion readmatrix(B : &vectix<vectix<intix>[3]>[3]) {
    forix (vectix<intix>[3] fila : B) {
        forix (intix elem : fila) {
            tabellae(elem);
        };
    };
}

potion printmatrix(C : vectix<vectix<intix>[3]>[3]) {
    forix (vectix<intix>[3] fila : B) {
        forix (intix elem : fila) {
            stilus(elem);
        };
    };
}

potion isInside(valor : intix, C : vectix<vectix<intix>[3]>[3]) -> boolix {
    boolix pertenece = romano; 
    forix (vectix<intix>[3] fila : B) {
        forix (intix elem : fila) {
            if (elem == valor) {
                    pertenece = galo;
            };
        };
    };
    return pertenece;
}

potion multmatrix(A : vectix<vectix<intix>[3]>[3], 
       B : vectix<vectix<intix>[3]>[3], C : &vectix<vectix<intix>[3]>[3]) {
    intix i = 0;
    intix j = 0;
    intix k = 0;
    whilix (i < 3) {
        whilix (j < 3) {
            whilix (k < 3) {
                C[i][j] = C[i][j] + A[i][k]*B[k][j];
                k = k+1;
            };
            j = j+1;
        };
        i = i+1;
    };
}

potion panoramix() -> intix {
    vectix<vectix<intix>[3]>[3] B;
    vectix<vectix<intix>[3]>[3] C; // C = A * B
    
    // Leemos la matriz B por consola
    readmatrix(B);
    
    // Calculamos el producto
    multmatrix(A, B, C);

    // Escribimos el resultado por consola
    printmatrix(C);

    // Vemos si el 4 esta en la matriz C
    stilus(isInside(4,C));
    
    return 0;
}
    \end{verbatim}
\end{document}
